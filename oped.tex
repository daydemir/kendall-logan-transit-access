\documentclass[11pt]{article}
\usepackage[margin=1in]{geometry}
\usepackage[T1]{fontenc}
\usepackage{cochineal}
\usepackage{titlesec}
\usepackage[colorlinks=true,linkcolor=blue,citecolor=blue,urlcolor=blue]{hyperref}
\usepackage{graphicx}
\usepackage{booktabs}
\usepackage{enumitem}
\usepackage{xcolor}
\usepackage{parskip}
\setlength{\parskip}{1em}

\title{\large Op-Ed\\\huge Can Massachusetts still pioneer?}
\author{Deniz Aydemir}
\date{December 8, 2025}

\begin{document}

\maketitle

Nobody likes going to the airport.

It's called luggage for a reason -- we have to lug it. If we want to drive ourselves, then we deal with exorbitant parking rates. If we want to take an Uber or Lyft then we don't know what it's going to cost until the day of our trip. And the traffic to get to the airport just keeps getting worse anyway.

We could build better transit to the airport, and this would be great. But we have a bigger problem. Logan is full.

Boston Logan Airport was constructed over 100 years ago. Last year, it hosted 43 million travelers, and that's growing at 2--3\% per year. By 2040, we'll be experiencing delays not because of weather, but because of crowding.

We're going to need a bigger airport. But that's still not the hardest problem. Logan is underwater.

Massachusetts needs \$7--15 billion by 2050 just for coastal protection, and Logan will be a major part of that bill. The increasing frequency of extreme weather events will also make it harder to land and take off at an airport that's on the shore, right where the storms hit.

We might not believe we need it today, and it may sound daunting to consider, but we have to face the truth. Boston will need a new airport, and we're going to need it in the next 30 years.

When we're ready to face the music, we need to learn our lessons from Logan. The next Boston airport should not take up ecologically sensitive waterfront real estate. The next Boston airport should not displace existing communities. The next Boston airport must be connected with high-speed electric rail connections to downtown Boston and MBTA metro lines.

These challenges present a new opportunity for Boston: the first U.S. airport built for rail connections.

By picking a location in the outskirts of Greater Boston, the next airport for this region can be an airport for all of Massachusetts. Electrified rail with level boarding increases speeds by up to 40\% compared with diesel-powered rail. We can find a location that connects Boston, Worcester, and Springfield to an international airport all within a one-hour train ride.

Boston has an opportunity to innovate on American airport design. Massachusetts is smaller than the Netherlands, which serves all its residents with rail-connected airports. There's no reason why our next airport can't be train-first. Airport traffic needs to become an artifact of the past.

We innovated on healthcare, now it's time to innovate on rail connectivity.

This is not only an opportunity for a new airport, but also for East Boston. The airport depresses land values and emits harmful particulates. East Boston children are four times more likely to have asthma.

To do right by East Boston, we need to begin undoing the damage Logan Airport wrought.

But the benefits of reclaiming the airport footprint will not just flow to residents of East Boston. All of Boston will unlock new opportunities for housing and development. Boston housing is some of the most expensive in the country, and we need to build more to meet demand. Adapting the airport's land for high-capacity housing and waterfront commercial development presents billions of dollars in potential revenue for the city, and a chance to effectively manage the fast-paced growth we're experiencing. And relocating airport traffic would significantly reduce congestion throughout the city.

We may not be ready today. Maybe not tomorrow. But Boston needs a new airport -- Massachusetts needs a new airport. When the time comes, we'll see if we still have that pioneering spirit. What do you think of Amelia Earhart International?


{\small \textit{This op-ed is adapted from a policy memo provided to the Kendall Square Association for ways to improve access to Boston Logan Airport. Here we focus on the bigger picture, what the future of air travel should look like for Boston, and for Massachusetts as a whole.}}


\end{document}
